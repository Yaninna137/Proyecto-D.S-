\documentclass {article}
\usepackage[spanish]{babel}
\usepackage{enumitem}
\usepackage[utf8]{inputenc}
\usepackage[letterpaper,top=2cm,bottom=2cm,left=3cm,right=3cm,marginparwidth=1.75cm]{geometry}
\usepackage[T1]{fontenc}
\usepackage{babel}
\usepackage{amsmath}
\usepackage{graphicx}
\usepackage{tabularx}
\usepackage[colorlinks=true, allcolors=blue]{hyperref}

\title{Titulo del proyecto D.S}
\author{Integrante 1\\Integrante 2\\Integrante 3\\Integrante 4}

\begin{document}
\maketitle
% Pagina 2 Docummento
\newpage
\section*{Ficha del Documento} 

\begin{table}[h!]
  \centering
  \begin{tabular}{|c|c|c|c|}
    \hline
    \textbf{Fecha} & \textbf{Revisión} & \textbf{Autor} & \textbf{Verificado dep. calidad} \\
    \hline
    \multicolumn{1}{|l|}{Texto fila 2 columna 1} & \multicolumn{1}{l|}{Texto fila 2 columna 2} & \multicolumn{1}{l|}{Texto fila 2 columna 3} & \multicolumn{1}{l|}{Texto fila 2 columna 4} \\
    \hline
  \end{tabular}
\end{table}
Documento validado por las partes en fechas: [fecha]
\begin{table}[h!]
  \centering
  \begin{tabular}{|l|l|}
    \hline
    \textbf{Por el cliente} & \textbf{Por la empresa suministradora} \\
    \hline
    Texto fila 2 columna 1 & Texto fila 2 columna 2 \\ \hline
    F.do. D/Dña [nombre] & F.do. D/Dña [nombre] \\ \hline
  \end{tabular}
\end{table}
\paragraph{}
% Pagina 3 Docummento
\newpage
\begin{center} % Centra el contenido.
{\Huge Contenido} % Título grande.
\end{center}
\vspace{1cm} % Espacio vertical.
{FICHAS DEL DOCUMENTO\\}
{CONTENIDO}
\begin{enumerate}
  \item INTRODUCCIÓN
    \begin{enumerate}[label*=\arabic*.]
      \item Propósito
      \item Alcance
      \item Personal involucrado
      \item Definiciones, acrónimos y abreviaturas
      \item Referencias
      \item Resumen
    \end{enumerate}
  \item DESCRIPCIÓN GENERAL
    \begin{enumerate}[label*=\arabic*.]
      \item Perspectiva del producto
      \item Funcionalidad del producto
    \end{enumerate}
\end{enumerate}
% Pagina 4 Documento
\newpage
\section{INTRODUCCIÓN}
Your introduction goes here! Simply start writing your document and use the Recompile button to view the updated PDF preview. Examples of commonly used commands and features are listed below, to help you get started.

Once you're familiar with the editor, you can find various project settings in the Overleaf menu, accessed via the button in the very top left of the editor. To view tutorials, user guides, and further documentation, please visit our \href{https://www.overleaf.com/learn}{help library}, or head to our plans page to \href{https://www.overleaf.com/user/subscription/plans}{choose your plan}.
\subsection{Propósito}
Your introduction goes here! Simply start writing your document and use the Recompile button to view the updated PDF preview. Examples of commonly used commands and features are listed below, to help you get started.

Once you're familiar with the editor, you can find various project settings in the Overleaf menu, accessed via the button in the very top left of the editor. To view tutorials, user guides, and further documentation, please visit our \href{https://www.overleaf.com/learn}{help library}, or head to our plans page to \href{https://www.overleaf.com/user/subscription/plans}{choose your plan}.
\subsection{Alcance}
Your introduction goes here! Simply start writing your document and use the Recompile button to view the updated PDF preview. Examples of commonly used commands and features are listed below, to help you get started.

Once you're familiar with the editor, you can find various project settings in the Overleaf menu, accessed via the button in the very top left of the editor. To view tutorials, user guides, and further documentation, please visit our \href{https://www.overleaf.com/learn}{help library}, or head to our plans page to \href{https://www.overleaf.com/user/subscription/plans}{choose your plan}.
\subsection{Personal involucrado}
\begin{tabularx}{\textwidth}{|>{\hsize=0.5\hsize}X|>{\hsize=1.5\hsize}X|}
\hline
 Columna 1 & Columna 2 \\
 \hline
 Nombre & Fila 1 \\
 \hline
 Rol & Fila 2 \\
 \hline
 Categoría profesional 3 & Fila 3 \\
 \hline
 Responsabilidad & Fila 4 \\
 \hline
 Información de contacto & Fila 5 \\
 \hline
 Aprobación & Fila 6 \\
 \hline
\end{tabularx}

\subsection{Definiciones, acrónimos y abreviaturas}
Your introduction goes here! Simply start writing your document and use the Recompile button to view the updated PDF preview. Examples of commonly used commands and features are listed below, to help you get started.

Once you're familiar with the editor, you can find various project settings in the Overleaf menu, accessed via the button in the very top left of the editor. To view tutorials, user guides, and further documentation, please visit our \href{https://www.overleaf.com/learn}{help library}, or head to our plans page to \href{https://www.overleaf.com/user/subscription/plans}{choose your plan}.
\subsection{Referencias}

\begin{tabularx}{\textwidth}{|X|X|X|X|X|}
\hline
\multicolumn{1}{|c|}{Referencia} & \multicolumn{1}{c|}{Titulo} & \multicolumn{1}{c|}{Ruta} & \multicolumn{1}{c|}{Fechas} & \multicolumn{1}{c|}{Autor} \\
\hline
\multicolumn{1}{|l|}{Fila 2} & \multicolumn{1}{l|}{Fila 2} & \multicolumn{1}{l|}{Fila 2} & \multicolumn{1}{l|}{Fila 2} & \multicolumn{1}{l|}{Fila 2} \\
\hline
\multicolumn{1}{|l|}{Fila 3} & \multicolumn{1}{l|}{Fila 3} & \multicolumn{1}{l|}{Fila 3} & \multicolumn{1}{l|}{Fila 3} & \multicolumn{1}{l|}{Fila 3} \\
\hline
\end{tabularx}

\subsection{Resumen}
Your introduction goes here! Simply start writing your document and use the Recompile button to view the updated PDF preview. Examples of commonly used commands and features are listed below, to help you get started.

Once you're familiar with the editor, you can find various project settings in the Overleaf menu, accessed via the button in the very top left of the editor. To view tutorials, user guides, and further documentation, please visit our \href{https://www.overleaf.com/learn}{help library}, or head to our plans page to \href{https://www.overleaf.com/user/subscription/plans}{choose your plan}.

% Pagina siguiente
\newpage
\section{DESCRIPCIÓN GENERAL}
Your introduction goes here! Simply start writing your document and use the Recompile button to view the updated PDF preview. Examples of commonly used commands and features are listed below, to help you get started.

Once you're familiar with the editor, you can find various project settings in the Overleaf menu, accessed via the button in the very top left of the editor. To view tutorials, user guides, and further documentation, please visit our \href{https://www.overleaf.com/learn}{help library}, or head to our plans page to \href{https://www.overleaf.com/user/subscription/plans}{choose your plan}.
\subsection{Perspectiva del producto}
Your introduction goes here! Simply start writing your document and use the Recompile button to view the updated PDF preview. Examples of commonly used commands and features are listed below, to help you get started.

Once you're familiar with the editor, you can find various project settings in the Overleaf menu, accessed via the button in the very top left of the editor. To view tutorials, user guides, and further documentation, please visit our \href{https://www.overleaf.com/learn}{help library}, or head to our plans page to \href{https://www.overleaf.com/user/subscription/plans}{choose your plan}.
\subsection{Funcionalidad del producto}
Your introduction goes here! Simply start writing your document and use the Recompile button to view the updated PDF preview. Examples of commonly used commands and features are listed below, to help you get started.

Once you're familiar with the editor, you can find various project settings in the Overleaf menu, accessed via the button in the very top left of the editor. To view tutorials, user guides, and further documentation, please visit our \href{https://www.overleaf.com/learn}{help library}, or head to our plans page to \href{https://www.overleaf.com/user/subscription/plans}{choose your plan}.

\end{document}
